\documentclass{ctexart}

\usepackage{amsmath}    % import \dfrac
\usepackage{amsthm}     % 引入定理环境:包括了证明 proof
\usepackage{minted}
% compiling with option `--shell-escape`
\setminted[c]{linenos=true, numberfirstline=true, frame=lines, autogobble, baselinestretch=0.8}
\usepackage{geometry}
\usepackage[colorlinks=true]{hyperref}
\usepackage{fancyhdr}
\pagestyle{fancy}
\fancyfoot{}            % 清除原来页脚
\fancyfoot[L]{项目地址:https://github.com/leetking/code-for-kaoyan}
\fancyfoot[R]{\thepage}

%\usemintedstyle{colorful}
% TODO mdify \emph style
%\DeclareTextFontCommand{\emph}{\textbf}


\title{数据结构和算法代码汇总}
\author{$\alpha$lpha0x00\\leetiankai@gmail.com}

\begin{document}
\maketitle

本文档来自于对 HarvestWu 的《数据结构算法汇总》文档的整理和完善,代码严格按照 C99 标准来书写,并不会 C++ 代码的任何特性(包括引用)。
\newpage

% need double compilations
\tableofcontents
\newpage

\section{线性表}

\subsection{数据结构定义}
\subsubsection{顺序表}
\paragraph{静态分配} 静态分配是指初始化链表时内部不需要 \mintinline{c}{malloc} 函数动态分配空间。其结构定义如下:
\begin{minted}{c}
#define MaxSize 1024
typedef struct SeqList {
    int data[MaxSize];
    int length; /* 记录存储元素的个数 */
} SeqList;
\end{minted}

\paragraph{动态分配} 相对于静态分配,初始化链表时需要 \mintinline{c}{malloc} 或 \mintinline{c}{calloc} 函数。
\begin{minted}{c}
typedef struct SeqList {
    int *data;
    int length, capacity;   /* capacity 记录顺序表可存储元素最大个数 */
} SeqList;
\end{minted}

\paragraph{基本操作}
\begin{enumerate}
\item 初始化 \mintinline{c}{bool InitList(SeqList *list, int capacity);} 成功返回 \mintinline{c}{true},失败返回 \mintinline{c}{false}
\begin{minted}{c}
/**
 * bool 在 C99 以标准类型添加到 C 语言中,来自头文件 <stdbool.h>
 */
bool InitList(SeqList *list, int capacity)
{
    if (NULL == list || capacity <= 0) return false;

    /**
     * 注意:这里没有使用
     * list->data = (int *)malloc(capacity * sizeof(int));
     * 风格,强制转换指针为 int * 类型,原因见下面链接
     * https://stackoverflow.com/questions/605845/do-i-cast-the-result-of-malloc
     */
    list->data = malloc(capacity * sizeof(int));
    if (NULL == list->data) return false;

    list->length = 0;
    list->capacity = capacity;
    return true;
}

/* 如果是静态分配的顺序表,初始化函数可以按照如下定义 */
bool InitList(SeqList *list)
{
    if (NULL == list) return false;
    list->length = 0;
    return true;
}
\end{minted}

\item 销毁列表 \mintinline{c}{void DestroyList(SeqList *list);} 对于静态顺序表,可以没有销毁操作,或者简单地把 \mintinline{c}{list->length} 置为 0 即可
\begin{minted}{c}
void DestroyList(SeqList *list)
{
    free(list->data);
}
\end{minted}

\item 求顺序表长度 \mintinline{c}{int ListLength(SeqList *list);}
\begin{minted}{c}
int ListLength(SeqList *list)
{
    if (NULL == list) return 0;
    return list->length;
}
\end{minted}

\item 获取顺序表容量 \mintinline{c}{int ListCapacity(SeqList *list);}
\begin{minted}{c}
int ListCapacity(SeqList *list)
{
    if (NULL == list) return 0;
    return list->capacity;
    /* 对于静态顺序表 */
    // return MaxSize;
}
\end{minted}

\item 顺序表是否为空 \mintinline{c}{bool ListEmpty(SeqList *list);}
\begin{minted}{c}
bool ListEmpty(SeqList *list)
{
    return (0 == list->length);
}
\end{minted}

\item 顺序表是否已满 \mintinline{c}{bool ListFull(SeqList *list);}
\begin{minted}{c}
bool ListFull(SeqList *list)
{
    return (ListLength(list) == ListCapacity(list));
}
\end{minted}

\item 查找某个元素的下标 \mintinline{c}{int LocateElem(SeqList *list, int ele);}
\begin{minted}{c}
/**
 * 返回 -1 表明没有找到或者出错
 */
int LocateElem(SeqList *list, int ele)
{
    if (NULL == list) return -1;
    for (int i = 0; i < Length(list); i++) {
        if (ele == list->data[i])
            return i;
        /* 对于静态顺序表 */
        //if (ele == list.data[i])
        //    return i;
    }
    /* 没有找到 */
    return -1;
}
\end{minted}

\item 根据下标获取元素 \mintinline{c}{int GetElem(SeqList *list, int idx);}
\begin{minted}{c}
int GetElem(SeqList *list, int idx)
{
    /* 假定元素都是大于 0,返回 -1 表明发生下标越界错误
     * 如果存储的元素是任意整数,那么返回 -1 就不合理,可以修改 GetElem 函数为
     * bool GetElem(SeqList *list, int idx, int *store);
     * 返回 ture,表明正确获取数据,存储到指针 *store 里,
     * 返回 false,表明下标越界
     */
    if (idx < 0 || idx > ListLength(list))
        return -1;
    return list->data[idx];
    /* 对于静态顺序表 */
    //return list.data[idx];
}
\end{minted}

\item 在指定下标处插入元素 \mintinline{c}{bool ListInsert(SeqList *list, int idx, int ele);}
\begin{minted}{c}
bool ListInsert(SeqList *list, int idx, int ele)
{
    if (idx < 0 || idx > ListLength(list)) return false;
    /* 存储空间已满 */
    if (ListFull(list)) return false;
    for (int i = Length(list); i > idx; i--) {
        list->data[i] = list->data[i-1];
        /* 对于静态顺序表 */
        // list.data[i] = list.data[i-1];
    }
    list->data[idx] = ele;
    return true;
}
\end{minted}

\item 删除下标处的元素 \mintinline{c}{bool ListDelete(SeqList *list, int idx, int *ele);}
\begin{minted}{c}
bool ListDelete(SeqList *list, int idx, int *ele)
{
    if (idx < 0 || idx > ListLength(list)) return false;
    for (int i = idx; i < ListLength(list)-1; i++) {
        list->data[i] = list->data[i+1];
        /* 对于静态顺序表 */
        // list.data[i] = list.data[i+1];
    }
    list->length = list->length-1;
    return true;
}
\end{minted}
\end{enumerate}

\subsubsection{单链表}
\begin{minted}{c}
typedef struct LinkNode {
    int data;
    struct LinkNode *next;
} LinkNode;
typedef LinkNode *LinkList;

/* 分配一个节点 */
LinkList node = malloc(sizeof(LinkNode));
node->next = NULL;
\end{minted}

\paragraph{基本操作}
\begin{enumerate}
\item 初始化一个单链表 \mintinline{c}{bool InitList(LinkList *list);}
\begin{minted}{c}
/**
 * 带头节点的初始化
 * LinkList 实际上是指向 LinkNode 类型的指针
 * 所以 list 指针也就是一个指向 LinkNode 类型的 “二级指针”,
 * 也就是所谓的指向指针的指针, list 指向 LinkList 类型的指针
 */
bool InitList(LinkList *list)
{
    if (NULL == list) return false;
    *list = malloc(sizeof(LinkNode));
    (*list)->next = NULL;
    return true;
}

/**
 * 不带头结点的单链表初始化
 */
 bool InitList(LinkList *list)
 {
    if (NULL == list) return false;
    *list = NULL;
    return true;
 }
 /* 对于不带头节点的单链表可以不使用函数来初始化链表,而是按照如下方式 */
 LinkList list = NULL;     /* 因为无头节点,可以直接以空指针当作空链表 */
\end{minted}

\item 释放一个单链表 \mintinline{c}{void DestroyList(LinkList list);}
\begin{minted}{c}
/**
 * 无论有没有头节点,释放方法都一样
 */
 void DestroyList(LinkList list)
 {
    LinkList next;
    for (; NULL != list; list = next) {
        next = list->next;
        free(list);
    }
}
\end{minted}

\end{enumerate}

\subsubsection{双链表}
双链表不像单链表有带头节点和不带头节点之分,因为不带头节点处理链表通常会繁琐一些,双链表插入删除涉及的指针数量是单链表的两倍,所以双链表实际中一般都是带有头节点。此外,双链表有普通的双链表和循环双链表两种实现。
\begin{minted}{c}
typedef struct DLinkNode {
    int data;
    struct DLinkNode *prev, *next;
} DLinkNode;
typedef DLinkNode *DLinkList;
\end{minted}

\paragraph{基本操作} 这里以\emph{循环双链}表为主要例子作为演示,同时注释里会注明普通双链表的实现。
\begin{enumerate}
\item 初始化双链表 \mintinline{c}{bool InitDLinkList(DLinkList *dlist);}
\begin{minted}{c}
bool InitDLinkList(DLinkList *dlist)
{
    if (NULL == dlist) return false;
    *dlist = malloc(sizeof(DLinkNode));
    if (NULL == *dlist) return false;
    /* 循环双链表 */
    (*dlist)->prev = (*dlist)->next = *dlist;
    /* 普通双链表 */
    // (*dlist)->prev = (*dlist)->next = NULL;
    return true;
}
/* 使用初始化函数 */
DLinkList list;
InitDLinkList(&list);
\end{minted}

\item 判断是否是空链表 \mintinline{c}{bool DLinkListEmpty(DLinkList list);}
\begin{minted}{c}
bool DLinkListEmpty(DLinkList list)
{
    if (NULL == list) return true;
    return (list->next = list) && (list->prev = list);
    /* 普通双链表 */
    // return (list->next == NULL) && (list->prev == NULL);
}
\end{minted}

\item 求表长度 \mintinline{c}{int DLinkListLength(DLinkList list);}
\begin{minted}{c}
int DLinkListLength(DLinkList list)
{
    if (NULL == list) return 0;
    int len = 0;
    /* 让 p = list->next 是为了去掉头节点 */
    for (DLinkList p = list->next; list != p; p = p->next)
        len++;

    /* 普通双链表 */
    //for (DLinkList p = list->next; NULL != p; p = p->next)
    //    len++;

    return len;
}
\end{minted}

\item 在双链表指定\emph{节点后}插入元素 \mintinline{c}{bool InsertDLinkListAt(DLinkList at, int ele);}
\begin{minted}{c}
/**
 * 插入元素 ele 到插入点 at 之后
 */
bool InsertDLinkListAt(DLinkList at, int ele)
{
    if (NULL == at) return false;
    DLinkNode *node = malloc(sizeof(DLinkNode));
    if (NULL == node) return false;
    node->data = ele;

    DLinkList next = at->next;
    node->next = next;
    node->prev = at;
    next->prev = node;
    at->next = node;
    return true;
}
\end{minted}

\item 删除双链表指定节点 \mintinline{c}{bool DeleteDLinkList(DLinkList node);}
\begin{minted}{c}
bool DeleteDLinkList(DLinkList node)
{
    if (NULL == node) return false;
    DLinkList prev = node->prev;
    DLinkList next = node->next;
    prev->next = next;
    next->prev = prev;
    free(node);
    return true;
}
\end{minted}
\end{enumerate}

\subsubsection{静态链表}
静态链表(或者叫「游标」的技术)在 C 语言中使用的比较少,但是对于没有指针概念的编程语言中使用会使用。与之类似的技术在 C 语言中使用较多,叫「池」的技术。比如链表节点池:为了减少 \mintinline{c}{malloc} 函数的调用以提高效率,预先分配一大块内存,自己在这块内存中分配各个节点。指向这些节点的时候可以不使用指针,而是使用节点在这一大块内存中的编号。

\subsection{例题}
\begin{enumerate}
    \item 删除不带头结点单链表 \mintinline{c}{list} 中所有值为 \mintinline{c}{x} 的结点
        \inputminted{c}{codes/delete-all-x-from-list.c}

    \item 删除带头结点单链表 \mintinline{c}{list} 中所有值为 \mintinline{c}{x} 结点
        \inputminted{c}{codes/delete-all-x-from-list2.c}

    \item 反向输出带有头结点单链表 \mintinline{c}{list} 中的所有值
        \inputminted{c}{codes/print-a-list-revsersely.c}

    \item 删除带头结点单链表 \mintinline{c}{list} 中最小值的节点
        \inputminted{c}{codes/delete-minimal-value-from-list.c}

    \item 就地逆置带头结点单链表 \mintinline{c}{list}
        \inputminted{c}{codes/reverse-list-in-place.c}

    \item 排序带头结点的单链表 \mintinline{c}{list} 为升序
        \inputminted{c}{codes/sort-list.c}

    \item 删除带有头结点单链表 \mintinline{c}{list} 中介于给定的两个值 \mintinline{c}{lo}, \mintinline{c}{hi} (大于等于 \mintinline{c}{lo},小于等于 \mintinline{c}{hi})中的元素
        \inputminted{c}{codes/delete-values-between-lo-hi-from-list.c}
\end{enumerate}

\section{栈}

\subsection{数据结构定义}
栈可以采用数组连续存储或者单链表来实现,其中使用动态分配的顺序存储来实现更为常见。下面按照动态分配的连续存储实现的栈演示。

\subsubsection{顺序存储的栈}
\paragraph{静态分配} 类似于静态分配的顺序表。
\begin{minted}{c}
#define MaxSize 1024
typedef struct SeqStack {
    int data[MaxSize];
    int top;
} SeqStack;
\end{minted}

\paragraph{动态分配}
\begin{minted}{c}
typedef struct SeqStack {
    int *data;
    int top, capacity;
} SeqStack;
\end{minted}

\paragraph{基本操作}
\begin{enumerate}
    \item 以最大容量 \mintinline{c}{capacity} 初始化栈 \mintinline{c}{bool InitStack(SeqStack *stack, int capacity);}
        \begin{minted}{c}
        bool InitStack(SeqStack *stack, int capacity)
        {
            if (NULL == stack || capacity <= 0) return false;
            stack->top = -1;
            stack->capacity = capacity;
            stack->data = malloc(capacity * sizeof(int));
            if (NULL == stack->data) return false;
            return true;
        }
        \end{minted}

    \item 销毁栈 \mintinline{c}{void DestroyStack(SeqStack *stack);} 和销毁动态分配的顺序表类似,就不再赘述

    \item 获取栈内元素个数 \mintinline{c}{int StackSize(SeqStack *stack);}
        \begin{minted}{c}
        int StackSize(SeqStack *stack)
        {
            if (NULL == stack) return 0;
            return (stack->top + 1);
        }
        \end{minted}

    \item 获取栈最大容量 \mintinline{c}{int StackCapacity(SeqStack *stack);}
        \begin{minted}{c}
        int StackCapacity(SeqStack *stack)
        {
            if (NULL == stack) return 0;
            return stack->capacity;
            /* 对于静态分配的栈 */
            // return MaxSize;
        }
        \end{minted}

    \item 判断栈是否为空 \mintinline{c}{bool StackEmpty(SeqStack *stack);}
        \begin{minted}{c}
        bool StackEmpty(SeqStack *stack)
        {
            if (NULL == stack) return true;
            return (0 == StackSize(stack));
            /* 或者 */
            // return  (-1 == stack->top);
        }
        \end{minted}

    \item 判断栈是否为满 \mintinline{c}{bool StackFull(SeqStack *stack);}
        \begin{minted}{c}
        bool StackFull(SeqStack *stack)
        {
            if (NULL == stack) return false;
            return (StackSize(stack) == StackCapacity(stack));
            /* 或者 */
            // return (stack->top +1 == stack->capacity);
        }
        \end{minted}

    \item 进栈 \mintinline{c}{bool Push(SeqStack *stack, int ele);}
        \begin{minted}{c}
        bool Push(SeqStack *stack, int ele)
        {
            if (NULL == stack || StackFull(stack)) return false;
            stack->top += 1;
            stack->data[stack->top] = ele;
            return true;
        }
        \end{minted}

    \item 出栈 \mintinline{c}{bool Pop(SeqStack *stack, int *store);}
        \begin{minted}{c}
        bool Pop(SeqStack *stack, int *store)
        {
            if (NULL == stack || NULL == store) return false;
            if (StackEmpty(stack)) return false;
            *store = stack->data[stack->top];
            stack->top -= 1;
            return true;
        }
        \end{minted}

    \item 获取栈顶元素 \mintinline{c}{bool GetTop(SeqStack *stack, int *store);}
        \begin{minted}{c}
        bool GetTop(SeqStack *stack, int *store)
        {
            if (NULL == stack || NULL == store) return false;
            if (StackEmpty(stack)) return false;
            *store = stack->data[stack->top];
            return true;
        }
        \end{minted}

\end{enumerate}

\subsubsection{链式存储的栈}
\begin{minted}{c}
typedef struct LinkStack {
    int size;   /* 存储了多少个元素 */
    LinkList top;
} LinkStack;
\end{minted}

\paragraph{基本操作}
\begin{enumerate}
\item 初始化栈 \mintinline{c}{bool InitStack(LinkStack *stack);}
\begin{minted}{c}
bool InitStack(LinkStack *stack)
{
    if (NULL === stack) return false;
    stack->size = 0;
    stack->top = NULL;  /* 不带头节点的单链表 */
    return true;
}
\end{minted}

\item 获取栈内元素个数、判断栈是否为空等函数不再赘述

\item 进栈 \mintinline{c}{bool Push(LinkStack *stack, int ele);}
\begin{minted}{c}
bool Push(LinkStack *stack, int ele)
{
    if (NULL == stack) return false;
    /* 链表实现的栈的容量为无穷大,所以不存在栈满的情况 */
    LinkNode *node = malloc(sizeof(LinkNode));
    if (NULL == node) return false;
    node->data = ele;

    node->next = stack->top;
    stack->top = node;

    stack->size += 1;
    return true;
}
\end{minted}

\item 出栈 \mintinline{c}{bool Pop(LinkStack *stack, int *store);}
\begin{minted}{c}
bool Pop(LinkStack *stack, int *store)
{
    if (NULL == stack || NULL == store) return false;
    if (Empty(stack)) return false;
    LinkList node = stack->top;
    stack->top = node->next;

    *store = node->data;
    free(node);
    return true;
}
\end{minted}

\item 获取栈顶元素 \mintinline{c}{int GetElem(LinkStack *stack);} 无须再给出吧
\end{enumerate}

\subsection{例题}
栈的应用更多是在数和图里面作为辅助数据结构使用,所以单独考察变体不多。这里以判断括号是否匹配为例子,演示栈数据结构的使用。
\inputminted{c}{codes/is-brackets-valid.c}


\section{队列}
\subsection{数据结构定义}
\subsubsection{顺序存储的队列}
同顺序表和顺序存储的栈,顺序存储的队列也有两种结构。下文只介绍动态分配的顺序存储队列的实现。
\paragraph{动态分配}
\begin{minted}{c}
typedef struct SeqQueue {
    int *data;
    int capacity;
    int front, rear;
    /* 通过 size 记录存储元素的个数,进而作为另一种判断队列是否为空的方法 */
    // int size;
} SeqQueue;
\end{minted}

\paragraph{基本操作}
\begin{enumerate}
\item 初始化队列 \mintinline{c}{bool InitQueue(SeqQueue *queue, int capacity);}
\begin{minted}{c}
bool InitQueue(SeqQueue *queue, int capacity)
{
    if (NULL == queue || capacity <= 0) return false;
    queue->data = malloc(capacity * sizeof(int));
    if (NULL == queue->data) return false;
    queue->capacity = capacity;
    queue->front = queue->rear = 0;
    // queue->size = 0;
    return true;
}
\end{minted}

\item 判断队列是否为空 \mintinline{c}{bool QueueEmpty(SeqQueue *queue);}
\begin{minted}{c}
bool QueueEmpty(SeqQueue *queue)
{
    if (NULL == queue) return true;
    return (queue->front == queue->rear);
    /* 如果采用 size 记录存储元素的个数 */
    // return (0 == queue->size);
}
\end{minted}

\item 判断队列是否为满 \mintinline{c}{bool QueueFull(SeqQueue *queue);}
\begin{minted}{c}
bool QueueFull(SeqQueue *queue)
{
    if (NULL == queue) return false;
    return (queue->front == (queue->rear+1) % queue->capacity);
    // return (queue->capacity == queue->size);
}
\end{minted}

\item 获取队列当前元素个数 \mintinline{c}{int QueueSize(SeqQueue *queue);}
\begin{minted}{c}
int QueueSize(SeqQueue *queue)
{
    if (NULL == queue) return 0;
    return (queue->rear - queue->front + queue->capacity) % queue->capacity;
    // return queue->size;
}
\end{minted}

\item 入队 \mintinline{c}{bool EnQueue(SeqQueue *queue, int ele);}
\begin{minted}{c}
bool EnQueue(SeqQueue *queue, int ele)
{
    if (NULL == queue || QueueFull(queue)) return false;
    queue->data[queue->rear] = ele;
    queue->rear = (queue->rear+1) % queue->capacity;
    // queue->size += 1;
    return true;
}
\end{minted}

\item 出队 \mintinline{c}{bool DeQueue(SeqQueue *queue, int *store);}
\begin{minted}{c}
bool DeQueue(SeqQueue *queue, int *store)
{
    if (NULL == queue || NULL == store) return false;
    if (QueueEmpty(queue)) return false;
    *store = queue->data[queue->front];
    queue->front = (queue->front+1) % queue->capacity;
    // queue->size -= 1;
    return true;
}
\end{minted}

\item 获取队首元素 \mintinline{c}{bool GetHead(SeqQueue *queue, int *store);}
\begin{minted}{c}
bool GetHead(SeqQueue *queue, int *store)
{
    if (NULL == queue || NULL == store) return false;
    if (QueueEmpty(queue)) return false;
    *store = queue->data[queue->front];
    return true;
}
\end{minted}
\end{enumerate}

\subsubsection{链式存储的队列}
单链表实现的队列,如果需要求得当前队列元素个数,必须要遍历队列,效率较低,也可以通过一个变量 \mintinline{c}{size} 记录存储元素的个数。
\begin{minted}{c}
typedef struct LinkQueue {
    LinkList rear, front;
    int size;
} LinkQueue;
\end{minted}

\paragraph{基本操作}
\begin{enumerate}
\item 初始化队列 \mintinline{c}{bool InitQueue(LinkQueue *queue);}
\begin{minted}{c}
bool InitQueue(LinkQueue *queue)
{
    if (NULL == queue) return false;
    LinkNode *dummy = malloc(sizeof(LinkNode)); /* 头节点 */
    if (NULL == dummy) return false;
    dummy->next = NULL;

    /**
     * front->[dummy]->NULL
     * rear ----^
     */
    queue->front = queue->rear = dummy;
    queue->size = 0;
    return true;
}
\end{minted}

\item 销毁队列 \mintinline{c}{void DestroyQueue(LinkQueue *queue);}
\begin{minted}{c}
void DestroyQueue(LinkQueue *queue)
{
    if (NULL == queue) return;
    DestroyList(queue->front);
}
\end{minted}

\item 判断队列是否为空 \mintinline{c}{bool QueueEmpty(LinkQueue *queue);}
\begin{minted}{c}
bool QueueEmpty(LinkQueue *queue)
{
    if (NULL == queue) return true;
    return (queue->front == queue->rear);
}
\end{minted}

\item 入队 \mintinline{c}{bool EnQueue(LinkQueue *queue, int ele);}
\begin{minted}{c}
bool EnQueue(LinkQueue *queue, int ele)
{
    if (NULL == queue) return false;
    LinkNode *node = malloc(sizeof(LinkNode));
    if (NULL == node) return false;
    node->data = ele;
    node->next = NULL;

    /**
     * front->[dummy]->[node1]->...->[ele]->NULL
     * rear ---------------------------^
     */
    queue->rear->next = node;
    queue->rear = node;

    queue->size += 1;
    return true;
}
\end{minted}

\item 出队 \mintinline{c}{bool DeQueue(LinkQueue *queue, int *store);}
\begin{minted}{c}
bool DeQueue(LinkQueue *queue, int *store)
{
    if (NULL == queue || NULL == store) return false;
    if (QueueEmpty(queue)) return false;

    /**
     * front->[dummy]->[node2]->...->[noden]->NULL
     * rear ---------------------------^
     * node ->[node1]
     */
    LinkList node = queue->front->next;
    queue->front->next = node->next;
    /* 如果是最后一个元素 */
    if (node == queue->rear)
        queue->rear = queue->front;
    queue->size -= 1;

    *store = node->data;
    free(node);
    return true;
}
\end{minted}
\end{enumerate}

\subsection{例题}
本章无例题:)


\section{树}
\subsection{数据结构定义}
\subsubsection{双亲存储结构}
\paragraph{数组形式}
数组形式存储的 $k\ (k \ge 2)$ 叉树,通常孩子节点下标 $n$ 和双亲节点下标 $m$ 满足如下关系(下标从 1 开始计数):
\begin{itemize}
    \item 通过孩子节点下标计算双亲节点下标:$m = \left\lceil \dfrac{n-1}{k} \right\rceil$
    \item 计算双亲节点的第 $i\ (1 \le i \le k)$ 个孩子的下标: $n = (m - 1)k+i+1$
\end{itemize}
下面证明如何通过孩子节点计算双亲节点公式正确:
\begin{proof}
    对于任意孩子节点 $n$,假设其双亲节点为 $m$。那么双亲 $m$ 的第一个孩子节点下标为 $(m-1) k+2$,最后一个孩子节点下标为 $mk+1$,那么有不等式
    $$
    \begin{aligned}
    & (m-1)k + 2 \le n \le mk+1 \\
        \implies & m -1 + \frac{2}{k} \le \frac{n}{k} \le m+\frac{1}{k} \\
        %\implies & m \le \frac{n}{k} + 1 - \frac{2}{k} \le m + 1 - \frac{1}{k} \\
        \implies & m+ \frac{1}{k} - 1 \le \frac{n-1}{k} \le m \\
        \implies & m \le \left\lceil \frac{n-1}{k} \right\rceil \le m \\
        \implies & m = \left\lceil \frac{n-1}{k} \right\rceil
    \end{aligned}
    $$
\end{proof}
特别地,对于二叉树上述公式也是成立,但是此外有更为便捷的方法。考虑到对计算机而言,下取整比上取整更为方便,对上述证明中采取如下计算方法:
$$
\begin{aligned}
& (m-1)k + 2 \le n \le mk+1 \\
    \implies & m \le \frac{n}{k} + 1 - \frac{2}{k} \le m + 1 - \frac{1}{k} \\
    \implies & m \le \left\lfloor \frac{n}{k} + 1 - \frac{2}{k} \right\rfloor \le \left\lfloor m + 1-\frac{1}{k} \right\rfloor = m \\
    \implies & m = \left\lfloor \frac{n}{k} + 1 - \frac{2}{k} \right\rfloor (\text{考虑到 } k = 2) \\
    \implies & m = \left\lfloor \frac{n}{2} \right\rfloor \\
\end{aligned}
$$
即由双亲节点得孩子节点 $n = 2m$ 或 $2m + 1$,由孩子节点得双亲节点 $m = \left\lfloor \dfrac{n}{2} \right\rfloor$

\begin{minted}{c}
#define TreeNodeMax 2048
/* Parent Tree Node */
typedef struct PTNode {
    int data;
    int parent;     /* -1 表明没有双亲节点 */
} PTNode;
/* Parent Tree */
typedef struct PTree {
    PTNode *nodes;
    int capacity;   /* nodes 的总个数 */
} PTree;
\end{minted}

\paragraph{应用}
\begin{enumerate}
    \item 并查集 (Union Find Set):顺序存储的树结构可以用于实现并查集,不同之处在于并查集不需要通过双亲节点找到孩子节点,因此可以更为灵活。注意:下列代码中空集定义为双亲元素为 -1,单元素集合定义为双亲元素为自身。
        \begin{itemize}
            \item 结构的定义
                \begin{minted}{c}
                typedef struct UFSet {
                    int *nodes;
                    int capacity;
                } UFSet;
                \end{minted}

            \item 初始化并查集 \mintinline{c}{bool InitUFSet(UFSet *set, int capacity);}
                \begin{minted}{c}
                bool InitUFSet(UFSet *set, int capacity)
                {
                    if (NULL == set || capacity <= 0) return false;
                    set->nodes = malloc(capacity * sizeof(int));
                    if (NULL == set->nodes) return false;
                    set->capacity = capacity;
                    /* 所有集合初始化为空集 */
                    for (int i = 0; i < capacity; i++)
                        set->nodes[i] = -1;
                    return true;
                }
                \end{minted}

            \item 销毁并查集 \mintinline{c}{void DestroyUFSet(UFSet *set);}
                \begin{minted}{c}
                void DestroyUFSet(UFSet *set)
                {
                    if (NULL == set) return;
                    free(set->nodes);
                }
                \end{minted}

            \item 添加单元素集合 \{id\} \mintinline{c}{bool MakeSet(UFSet *set, int id);}
                \begin{minted}{c}
                bool MakeSet(UFSet *set, int id)
                {
                    if (NULL == set) return false;
                    if (id < 0 || id >= set->capacity) return false;
                    /* 元素 id 已经添加 */
                    if (set->nodes[id] != -1) return false;
                    set->nodes[id] = id;
                    return true;
                }
                \end{minted}

            \item 查找元素 \mintinline{c}{id} 所在集合的代表元素 \mintinline{c}{int Find(UFSet *set, int id);}
                \begin{minted}{c}
                /**
                 * 返回 -1 表示 id 不在任何集合内
                 * 返回 >= 0 表示 id 所在集合的代表元素
                 */
                int Find(UFSet *set,  int id)
                {
                    if (NULL == set) return -1;
                    if (id < 0 || id >= set->capacity) return -1;
                    while (true) {
                        int parent = set->nodes[id];
                        if (parent == id || parent == -1)
                            return parent;
                        id = parent;
                    }
                }
                \end{minted}

            \item 判断两个元素是否同一集合 \mintinline{c}{bool Same(UFSet *set, int x, int y);}
                \begin{minted}{c}
                bool Same(UFSet *set, int x, int y)
                {
                    int xparent = Find(set, x);
                    int yparent = Find(set, y);
                    if (-1 == xparent || -1 == yparent)
                        return false;
                    return (xparent == yparent);
                }
                \end{minted}

            \item 合并 \mintinline{c}{x}  和 \mintinline{c}{y} 所在的集合 \mintinline{c}{bool Union(UFSet *set, int x, int y);}
                \begin{minted}{c}
                /**
                 * 下面单纯的合并会导致查找 Find 效率降低,
                 * 实际中会使用:按秩合并和路径压缩的方法
                 * 参见:https://en.wikipedia.org/wiki/Disjoint-set_data_structure
                 */
                 bool Union(UFSet *set, int x, int y)
                 {
                    int xparent = Find(set, x);
                    int yparent = Find(set, y);
                    if (-1 == xparent || -1 == yparent)
                        return false;
                    /* y 所在集合的代表元素 yparent 指向 x 集合代表元素 xparent */
                    set->nodes[yparent] = xparent;
                    return true;
                 }
                \end{minted}
        \end{itemize}
    \item 满二叉树
    \item 大堆或小堆 % TODO 实现大堆和小堆
\end{enumerate}

\paragraph{指针形式} 指针形式通常不能直接找到双亲节点的孩子节点,通过孩子节点寻找双亲节点很方便。
\begin{minted}{c}
typedef struct PTreeNode {
    int data;
    struct PTreeNode *parent;
} PTreeNode;
typedef PTreeNode *PTree;
\end{minted}

\subsubsection{孩子兄弟存储结构}
\begin{minted}{c}
/* Child Sibling Tree */
typedef struct CSTreeNode {
    int data;
    struct CSTreeNode *firstchild, *nextsibling;
} CSTreeNode;
typedef CSTreeNode *CSTree;
\end{minted}

\subsubsection{二叉树、二叉排序数和平衡二叉树}
%\paragraph{二叉树结构定义}
\begin{minted}{c}
/* Binary Tree */
typedef struct BiTreeNode {
    int data;
    struct BiTreeNode *lchild, *rchild;
} BiTreeNode;
typedef BiTreeNode *BiTree;
\end{minted}
\paragraph{基本操作}
\begin{enumerate}
    \item 二叉树遍历
        \begin{itemize}
            \item 先序遍历 \mintinline{c}{void PreOrder(BiTree tree);}
                \begin{minted}{c}
                /* 递归实现 */
                void PreOrder(BiTree tree)
                {
                    if (NULL == tree) return;
                    /* 访问节点 tree */
                    // visit(tree);
                    PreOrder(tree->lchild);
                    PreOrder(tree->rchild);
                }

                /* 非递归实现:版本一 */
                void PreOrder(BiTree tree)
                {
                #define TreeHeightMax 2048
                    SeqStack stack;
                    InitStack(&stack, TreeHeightMax);

                    Push(&stack, &tree);
                    while (!StackEmpty(&stack)) {
                        BiTree node;
                        Pop(&stack, &node);
                        if (NULL == node)
                            continue;
                        /* 访问节点 node */
                        // visit(node);
                        /* 先右节点入栈,然后是左节点入栈 */
                        Push(&stack, &node->rchild);
                        Push(&stack, &node->lchild);
                    }

                    DestroyStack(&stack);
                }

                /* 非递归实现:版本二 */
                void PreOrder(BiTree tree)
                {
                #define TreeHeightMax 2048
                    SeqStack stack;
                    InitStack(&stack, TreeHeightMax);

                    BiTree curr = tree;
                    while (NULL != curr || !StackEmpty(&stack)) {
                        if (NULL != curr) {
                            /* 访问当前节点 curr */
                            // visit(curr);
                            Push(&stack, &curr);
                            curr = curr->lchild;
                        } else {
                            Pop(&stack, &curr);
                            curr = curr->rchild;
                        }
                    }

                    DestroyStack(&stack);
                }
            \end{minted}

            \item 中序遍历 \mintinline{c}{void InOrder(BiTree tree);}
                \begin{minted}{c}
                /* 递归实现 */
                void InOrder(BiTree tree)
                {
                    if (NULL == tree) return;
                    InOrder(tree->lchild);
                    /* 访问节点 tree */
                    // visit(tree);
                    InOrder(tree->rchild);
                }

                /* 非递归实现 */
                void InOrder(BiTree tree)
                {
                #define TreeHeightMax 2048
                    SeqStack stack;
                    InitStack(&stack, TreeHeightMax);

                    BiTree curr = tree;
                    while (NULL == curr || !StackEmpty(&stack)) {
                        if (NULL != curr) {
                            Push(&stack, &curr);
                            curr = curr->lchild;
                        } else {
                            Pop(&stack, &curr);
                            /* 访问节点 node */
                            // visit(node);
                            curr = curr->rchild;
                        }
                    }

                    DestroyStack(&stack);
                }
                \end{minted}

            \item 后序遍历 \mintinline{c}{void PostOrder(BiTree tree);}
                \begin{minted}{c}
                /* 递归实现 */
                void PostOrder(BiTree tree)
                {
                    if (NULL == tree) return;
                    PostOrder(tree->lchild);
                    PostOrder(tree->rchild);
                    /* 访问节点 tree */
                    // visit(tree);
                }

                /* 非递归实现 */
                /* 参见 https://en.wikipedia.org/wiki/Tree_traversal#Post-order_(LRN) */
                void PostOrder(BiTree tree)
                {
                #define TreeHeightMax 2048
                    SeqStack stack;
                    InitStack(&stack, TreeHeightMax);
                    BiTree lastVisited = NULL;
                    BiTree curr = tree;

                    while (NULL != curr || !StackEmpty(&stack)) {
                        if (NULL != curr) {
                            Push(&stack, &curr);
                            curr = curr->lchild;
                        } else {
                            GetTop(&stack, &curr);  /* 只获取栈顶,不弹栈 */
                            if (NULL != curr->rchild && lastVisited != curr->rchild) {
                                curr = curr->rchild;
                            } else {
                                /* 访问当前节点 */
                                // visit(curr);
                                Pop(&stack, &lastVisited);
                            }
                        }
                    }

                    DestroyStack(&stack);
                }
                \end{minted}
            \item 层次遍历 \mintinline{c}{void LevelOrder(BiTree tree);}
                \begin{minted}{c}
                void LevelOrder(BiTree tree)
                {
                #define TreeWidthMax 4096
                    if (NULL == tree) return;
                    SeqQueue queue;
                    InitQueue(&queue, TreeNodesMax);

                    EnQueue(&queue, &tree);
                    while (!QueueEmpty(&queue)) {
                        BiTree node;
                        DeQueue(&queue, &node);
                        /* 访问节点 */
                        // visit(node);
                        if (NULL != node->lchild)
                            EnQueue(&queue, &node->lchild);
                        if (NULL != node->rchild)
                            EnQueue(&queue, &node->rchild);
                    }

                    DestroyQueue(&queue);
                }
                \end{minted}
        \end{itemize}

    \item 二叉查找树 (Binary Search Tree, BST),也叫排序二叉树或二叉排序树
        \begin{itemize}
            \item 查找元素 \mintinline{c}{bool BiTreeSearch(BiTree tree, int data);}
                \begin{minted}{c}
                /* 递归实现 */
                bool BiTreeSearch(BiTree tree, int data)
                {
                    if (NULL == tree) return false;
                    if (data == tree->data)
                        return true;
                    if (data < tree->data)
                        return BiTreeSearch(tree->lchild, data);
                    else
                        return BiTreeSearch(tree->rchild, data);
                }

                /* 非递归实现 */
                bool BiTreeSearch(BiTree tree, int data)
                {
                    while (NULL != tree) {
                        if (data == tree->data)
                            return true;
                        if (data < tree->lchild)
                            tree = tree->lchild;
                        else
                            tree = tree->rchild;
                    }

                    return false;
                }
                \end{minted}

            \item 插入元素 \mintinline{c}{bool BiTreeInsert(BiTree *ptree, int data);}
                \inputminted{c}{codes/bst-insert.c}

            \item 二叉排序树构造 \mintinline{c}{BiTree CreateBSTree(int *array, int n);}
                \begin{minted}{c}
                BiTree CreateBSTree(int *array, int n)
                {
                    if (NULL == array || n <= 0) return NULL;

                    BiTree tree = NULL;
                    for (int i = 0; i < n; i++)
                        BiTreeInsert(&tree, array[i]);

                    return tree;
                }
                \end{minted}

            \item 删除元素 \mintinline{c}{bool BiTreeDelete(BiTree *ptree, int data);}
                \inputminted{c}{codes/bst-delete.c}

        \end{itemize}

    \item 平衡二叉树 AVL-Tree,查找算法和二叉查找树一样,但是插入和删除算法不同而且代码实现较为复杂(比二叉查找树删除元素还要复杂),这里暂时不给出。
\end{enumerate}

\subsubsection{线索二叉树}
\begin{minted}{c}
typedef struct ThreadTreeNode {
    int data;
    struct ThreadTreeNode *lchild, *rchild;
#define TagChild    0
#define TagPrevNode 1
#define TagNextNode 1
    int ltag, rtag;
} ThreadTreeNode;
/**
 * ltag 为 TagChild    (0) 时,lchild 指向左孩子节点
 * ltag 为 TagPrevNode (1) 时,lchild 指向前驱节点
 * rtag 为 TagChild    (0) 时,rchild 指向右孩子结点
 * rtag 为 TagNextNode (1) 时,rchild 指向后继节点
 */
 typedef ThreadTreeNode *ThreadTree;
\end{minted}
思想:任意一个节点有前驱节点且左孩子指针 \mintinline{c}{lchild} 没有指向左孩子,那么就指向此节点的前驱节点;任意一个节点有后继节点且右孩子指针 \mintinline{c}{rchild} 没有指向右孩子,那么就指向后继节点。

\paragraph{基本操作}
\begin{enumerate}
    \item 先序遍历线索化 \mintinline{c}{void CreatePreThreadTree(ThreadTree tree);}
        \inputminted{c}{codes/create-preorder-thread-tree.c}

    \item 中序遍历线索化 \mintinline{c}{void CreateInThreadTree(ThreadTree tree);}
        \inputminted{c}{codes/create-inorder-thread-tree.c}

    \item 后序遍历线索化 \mintinline{c}{void CreatePostThreadTree(ThreadTree tree);}
        \inputminted{c}{codes/create-postorder-thread-tree.c}

    \item 先序遍历先序线索树 \mintinline{c}{void PreOrderOfPreThreadTree(ThreadTree tree);}
        \inputminted{c}{codes/preorder-of-prethread-tree.c}

    \item 中序遍历中序线索树 \mintinline{c}{void InOrderOfInThraedTree(ThreadTree tree);}
        \inputminted{c}{codes/inorder-of-inthread-tree.c}
        对中序线索树中序遍历时,某一节点的前驱或者后继节点一定是其\emph{子树}中的节点,所以一定方便找到前驱和后继。对比先序线索树的先序遍历,某一节点的前去节点可能是其\emph{父节点},所以难以找到前驱节点。

    \item 后序遍历后序线索树 \mintinline{c}{void PostOrderOfPostThread(ThreadTree tree);}
        \inputminted{c}{codes/postorder-of-postthread-tree.c}
        对比先序线索树的先序遍历,可以发现后序线索树的后续遍历与其有一定的对称性。
\end{enumerate}

\subsection{例题}
\begin{enumerate}
    \item 求二叉树的高度
        \inputminted{c}{codes/height-of-bitree.c}

    \item 求二叉树宽度和高度,二叉树宽度定义为同一层中节点数最大值
        \inputminted{c}{codes/height-and-width-of-bitree.c}

    \item 求左子树中节点经过根节点到右子树中节点的最长路径的长度
        \inputminted{c}{codes/max-distance-through-root.c}

    \item 输出根节点到每个叶子结点的路径
        \inputminted{c}{codes/print-all-traces.c}

    \item 输出二叉树给定节点的所有祖先
        \inputminted{c}{codes/print-ancestors.c}

    \item 判断给定二叉树是否为完全二叉树
        \inputminted{c}{codes/is-complete-bitree.c}

    \item 判断二叉树 \mintinline{c}{subtree} 是否为 \mintinline{c}{tree} 的一棵\emph{子树}
        \inputminted{c}{codes/is-subtree.c}

    \item 判断二叉树 \mintinline{c}{part} 是否为 \mintinline{c}{tree} 的\emph{子结构},子结构是指能在 \mintinline{c}{tree} 找到 \mintinline{c}{part} 这样的树结构,并不一定是子树。
        \inputminted{c}{codes/is-part-of-bitree.c}

    \item 按照从下往上、从右往左的方式遍历二叉树(逆层次遍历)
        \inputminted{c}{codes/re-level-order.c}

    \item 通过先序序列和中序序列重构二叉树
        \inputminted{c}{codes/rebuild-bitree.c}

    \item 由满二叉树的先序序列求出其后序序列,也就是对顺序存储的满二叉树数组进行后续遍历
        \inputminted{c}{codes/from-preorder-to-postorder.c}

    \item 统计二叉树的节点数、叶子数、双分叉节点数
        \inputminted{c}{codes/statistic-of-bitree.c}

    \item 交换左右子树
        \inputminted{c}{codes/swap-bitree.c}

    \item 删除二叉树所有值为 \mintinline{c}{x} 的\emph{子树}(不是节点)
        \inputminted{c}{codes/delete-all-subtree-x-from-bitree.c}

    \item 删除二叉树(不是二叉排序树)所有值为 \mintinline{c}{x} 的\emph{节点},由于不是二叉排序树,所以删除节点后的树不唯一,只需要保证是二叉树即可
        \inputminted{c}{codes/delete-all-x-from-bitree.c}

    \item 查找二叉树节点 \mintinline{c}{p} 和 \mintinline{c}{q} 的最近公共祖先
        \inputminted{c}{codes/common-ancestor.c}

    \item 将二叉树的叶子节点从左向右顺序连接,即把所有叶子节点组织成非循环双链表
        \inputminted{c}{codes/link-leafs-of-bitree.c}

    \item 判断两棵二叉树结构是否相似(只判断形状不判断节点值)
        \inputminted{c}{codes/is-same-bitree.c}

    \item 在中序线索树查找给定节点在\emph{后序遍历序列}中的前驱节点
        \inputminted{c}{codes/find-previous-node-of-postorder-in-inthreadtree.c}

    \item 计算二叉树的带权路径长度 (Weighted Path Length, WPL)
        \inputminted{c}{codes/weighted-path-length-of-bitree.c}

    \item 输出给定表达树的中缀表达式子(通过括号体现优先级)
        \inputminted{c}{codes/print-expression-bitree.c}

    \item 求指定值在二叉排序树中的高度
        \inputminted{c}{codes/height-of-value-in-bsttree.c}

    \item 判断给定二叉树是否是\emph{二叉排序树}
        \inputminted{c}{codes/is-bsttree.c}


    \item 判断给定二叉树是否是\emph{平衡二叉树}
        \inputminted{c}{codes/is-avl-tree.c}

    \item 求二叉排序树的最大值和最小值
        \inputminted{c}{codes/max-and-min-value-in-bsttree.c}

    \item \emph{从大到小} 输出二叉排序树中大于等于 \mintinline{c}{lo} 小于等于 \mintinline{c}{hi} 的节点的值
        \inputminted{c}{codes/print-from-hi-to-lo-in-bsttree.c}

    \item 以 $O(\log_2(n))$ 时间复杂度查找二叉树(假定二叉树树没有退化成链表)中序遍历序列中第 $k\ (1 \le k \le n)$ 个节点。二叉树节点中有附加域 \mintinline{c}{nodes} 表明当前子树的节点个数,比如空树没有节点,所以无法存储 \mintinline{c}{nodes},就认为空树节点数为 0;只有一个节点的子树其根节点 \mintinline{c}{nodes} 为 1;其他树的根节点的 \mintinline{c}{nodes} 等于左右子树的 \mintinline{c}{nodes} 之和再加 1
        \inputminted{c}{codes/find-kth-node-in-bitree.c}

\end{enumerate}


\section{广义表}
\subsection{数据结构定义}

\subsection{例题}



\section{图}
\subsection{数据结构定义}


\subsection{例题}


\section{排序和查找}


\end{document}
