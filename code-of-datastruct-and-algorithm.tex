\documentclass{ctexart}

\usepackage{amsmath}    % import \dfrac
\usepackage{amsthm}     % 引入定理环境:包括了证明 proof
\usepackage{minted}
% compiling with option `--shell-escape`
\setminted[c]{linenos=true, numberfirstline=true, frame=lines, autogobble, baselinestretch=0.8}
\usepackage{geometry}
\usepackage[colorlinks=true]{hyperref}
\usepackage{fancyhdr}
\pagestyle{fancy}
\fancyfoot{}            % 清除原来页脚
\fancyfoot[L]{项目地址:https://github.com/leetking/code-for-kaoyan}
\fancyfoot[R]{\thepage}

%\usemintedstyle{colorful}
% TODO mdify \emph style
%\DeclareTextFontCommand{\emph}{\textbf}


\title{数据结构和算法代码汇总}
\author{$\alpha$lpha0x00\\leetiankai@gmail.com}

\begin{document}
\maketitle

本文档来自于对 HarvestWu 的《数据结构算法汇总》文档的整理和完善,代码严格按照 C99 标准来书写,并不会 C++ 代码的任何特性(包括引用)。
\newpage

% need double compilations
\tableofcontents
\newpage

\section{线性表}

\subsection{数据结构定义}
\subsubsection{顺序表}
\paragraph{静态分配} 静态分配是指初始化链表时内部不需要来自头文件 \mintinline{c}{<stdlib.h>} 的 \mintinline{c}{malloc} 或 \mintinline{c}{calloc} 函数动态分配空间。其结构定义如下:
\begin{minted}{c}
#define MaxSize 1024
typedef struct SeqList {
    int data[MaxSize];
    int length; /* 记录存储元素的个数 */
} SeqList;
\end{minted}

\paragraph{动态分配} 相对于静态分配,初始化链表时需要 \mintinline{c}{malloc} 或 \mintinline{c}{calloc} 函数。
\begin{minted}{c}
typedef struct SeqList {
    int *data;
    int length, capacity;   /* capacity 记录顺序表可存储元素最大个数 */
} SeqList;
\end{minted}

\paragraph{基本操作}
\begin{enumerate}
    \item 初始化顺序表 \mintinline{c}{bool InitList(SeqList *list, int capacity);} 成功返回 \mintinline{c}{true},失败返回 \mintinline{c}{false}. 注意:下面没有使用 \mintinline{c}{list->data = (int *)malloc(capacity * sizeof(int))} 风格 ,即强制转换 \mintinline{c}{malloc} 返回值类型为 \mintinline{c}{int *} 类型,原因参见 \footnote{\url{https://stackoverflow.com/questions/605845/do-i-cast-the-result-of-malloc}}。
        \inputminted{c}{codes/init-static-list.c}
        如果是静态分配的顺序表,初始化函数可以按照如下定义:
        \begin{minted}{c}
        bool InitList(SeqList *list)
        {
            if (NULL == list) return false;
            list->length = 0;
            return true;
        }
        \end{minted}

    \item 销毁顺序表 \mintinline{c}{void DestroyList(SeqList *list);} 对于静态顺序表,可以没有销毁操作,或者简单地把 \mintinline{c}{list->length} 置为 0 即可
        \begin{minted}{c}
        void DestroyList(SeqList *list)
        {
            free(list->data);
        }
        \end{minted}

    \item 求顺序表长度 \mintinline{c}{int ListLength(SeqList *list);}
        \begin{minted}{c}
        int ListLength(SeqList *list)
        {
            if (NULL == list) return 0;
            return list->length;
        }
        \end{minted}

    \item 获取顺序表容量 \mintinline{c}{int ListCapacity(SeqList *list);}
        \begin{minted}{c}
        int ListCapacity(SeqList *list)
        {
            if (NULL == list) return 0;
            return list->capacity;
            /* 对于静态顺序表 */
            //return MaxSize;
        }
        \end{minted}

    \item 顺序表是否为空 \mintinline{c}{bool ListEmpty(SeqList *list);}
        \begin{minted}{c}
        bool ListEmpty(SeqList *list)
        {
            return (0 == list->length);
        }
        \end{minted}

    \item 顺序表是否已满 \mintinline{c}{bool ListFull(SeqList *list);}
        \begin{minted}{c}
        bool ListFull(SeqList *list)
        {
            return (ListLength(list) == ListCapacity(list));
        }
        \end{minted}

    \item 查找某个元素的下标 \mintinline{c}{int LocateElem(SeqList *list, int ele);}
        \inputminted{c}{codes/locate-element-in-static-list.c}

    \item 根据下标获取元素 \mintinline{c}{int GetElem(SeqList *list, int idx);}
        \inputminted{c}{codes/get-element-from-static-list.c}

    \item 在指定下标处插入元素 \mintinline{c}{bool ListInsert(SeqList *list, int idx, int ele);}
        \inputminted{c}{codes/static-list-insert.c}

    \item 删除下标处的元素 \mintinline{c}{bool ListDelete(SeqList *list, int idx, int *ele);}
        \inputminted{c}{codes/static-list-delete.c}
\end{enumerate}

\subsubsection{单链表}
\begin{minted}{c}
typedef struct LinkNode {
    int data;
    struct LinkNode *next;
} LinkNode;
typedef LinkNode *LinkList;

/* 分配一个节点 */
LinkList node = malloc(sizeof(LinkNode));
node->next = NULL;
\end{minted}

\paragraph{基本操作}
\begin{enumerate}
    \item 初始化一个单链表 \mintinline{c}{bool InitList(LinkList *list);}
        单链表常常有两种组织方式,带有头节点和不带头节点。带有头节点的单链表很多操作可以方便一下,不带头节点的单链表虽然也可以向带有头节点的一样,但是要稍稍繁琐一点。
        \begin{itemize}
            \item 带头节点的单链表初始化
                \inputminted{c}{codes/init-link-list-with-dummy-node.c}
            \item 不带头节点的单链表初始化
                \inputminted{c}{codes/init-link-list-without-dummy-node.c}
        \end{itemize}

    \item 销毁单链表,释放节点占据的空间 \mintinline{c}{void DestroyList(LinkList list);}
        \inputminted{c}{codes/destroy-link-list.c}

\end{enumerate}

\subsubsection{双链表}
双链表不像单链表有带头节点和不带头节点之分,因为不带头节点处理链表通常更繁琐一些,双链表插入删除涉及的指针数量是单链表的两倍,所以双链表实际中一般都是带有头节点。此外,双链表有普通的双链表和循环双链表两种实现。
\begin{minted}{c}
typedef struct DLinkNode {
    int data;
    struct DLinkNode *prev, *next;
} DLinkNode;
typedef DLinkNode *DLinkList;
\end{minted}

\paragraph{基本操作} 这里以\emph{循环双链}表为主要例子作为演示,同时注释里会注明普通双链表的实现。
\begin{enumerate}
    \item 初始化双链表 \mintinline{c}{bool InitDLinkList(DLinkList *dlist);}
        \inputminted{c}{codes/init-double-link-list.c}
        按照如下方式定义双链表并初始化:
        \begin{minted}{c}
        DLinkList list;         /* 定义双链表 list */
        InitDLinkList(&list);   /* 初始化双链表 list */
        \end{minted}

    \item 判断是否是空链表 \mintinline{c}{bool DLinkListEmpty(DLinkList list);}
        \begin{minted}{c}
        bool DLinkListEmpty(DLinkList list)
        {
            if (NULL == list) return true;
            return (list->next = list) && (list->prev = list);
            /* 普通双链表 */
            //return (list->next == NULL) && (list->prev == NULL);
        }
        \end{minted}

    \item 求表长度 \mintinline{c}{int DLinkListLength(DLinkList list);}
        \inputminted{c}{codes/length-of-double-link-list.c}

    \item 在双链表指定\emph{节点后}插入元素 \mintinline{c}{bool InsertDLinkListAt(DLinkList at, int ele);}
        \begin{itemize}
            \item 在循环双链表中插入元素
                \inputminted{c}{codes/insert-double-cycle-link-at.c}
            \item 在非循环双链表中插入元素
                \inputminted{c}{codes/insert-double-link-at.c}
        \end{itemize}

    \item 删除双链表中指定节点 \mintinline{c}{bool DeleteDLinkList(DLinkList node);}
        \begin{itemize}
            \item 从循环双链表中删除元素
                \inputminted{c}{codes/delete-in-double-cycle-link-list.c}
            \item 从非循环双链表中删除元素
                \inputminted{c}{codes/delete-in-double-link-list.c}
        \end{itemize}

\end{enumerate}

\subsubsection{静态链表}
静态链表(或者叫「游标」的技术)在 C 语言中使用的比较少,但是对于没有指针概念的编程语言中使用会使用。与之类似且在 C 语言中使用较多的技术,叫「池」。比如链表节点池:为了减少 \mintinline{c}{malloc} 函数的调用以提高效率,预先分配一大块内存,自己在这块内存中分配各个节点。指向这些节点的时候可以不使用指针,而是使用节点在这一大块节点池中的编号。

\subsection{例题}
\begin{enumerate}
    \item 删除不带头结点单链表 \mintinline{c}{list} 中所有值为 \mintinline{c}{x} 的结点
        \inputminted{c}{codes/delete-all-x-from-list.c}

    \item 删除\emph{带头结点}单链表 \mintinline{c}{list} 中所有值为 \mintinline{c}{x} 结点
        \inputminted{c}{codes/delete-all-x-from-list2.c}

    \item 反向输出\emph{带有头结点}单链表 \mintinline{c}{list} 中的所有值
        \begin{itemize}
            \item 方法一:函数递归输出,由于链表长度可以很长,所以存在栈溢出的问题
                \inputminted{c}{codes/print-list-revsersely1.c}
            \item 方法二:用一个数组暂存元素然后逆向输出(就是后面会讲到的栈的思想)
                \inputminted{c}{codes/print-list-revsersely2.c}
        \end{itemize}

    \item 删除\emph{带头结点}单链表 \mintinline{c}{list} 中最小值的节点
        \inputminted{c}{codes/delete-minimal-value-from-list.c}

    \item 就地逆置\emph{带头结点}单链表 \mintinline{c}{list}
        \inputminted{c}{codes/reverse-list-in-place.c}

    \item 排序\emph{带头结点}单链表 \mintinline{c}{list} 为升序
        \inputminted{c}{codes/sort-list.c}

    \item 删除\emph{带头结点}单链表 \mintinline{c}{list} 中介于给定的两个值 \mintinline{c}{lo}, \mintinline{c}{hi} (大于等于 \mintinline{c}{lo},小于等于 \mintinline{c}{hi})中的元素
        \inputminted{c}{codes/delete-values-between-lo-hi-from-list.c}
\end{enumerate}

\section{栈}

\subsection{数据结构定义}
栈可以采用数组连续存储或者单链表来实现,其中使用动态分配的顺序存储来实现更为常见。下面按照动态分配的连续存储实现的栈演示。

\subsubsection{顺序存储的栈}
\paragraph{静态分配} 类似于静态分配的顺序表。
\begin{minted}{c}
#define MaxSize 1024
typedef struct SeqStack {
    int data[MaxSize];
    int top;
} SeqStack;
\end{minted}

\paragraph{动态分配}
\begin{minted}{c}
typedef struct SeqStack {
    int *data;
    int top, capacity;
} SeqStack;
\end{minted}

\paragraph{基本操作}
\begin{enumerate}
    \item 以最大容量 \mintinline{c}{capacity} 初始化栈 \mintinline{c}{bool InitStack(SeqStack *stack, int capacity);}
        \inputminted{c}{codes/init-sequence-stack.c}

    \item 销毁栈 \mintinline{c}{void DestroyStack(SeqStack *stack);} 和销毁动态分配的顺序表类似,就不再赘述

    \item 获取栈内元素个数 \mintinline{c}{int StackSize(SeqStack *stack);}
        \begin{minted}{c}
        int StackSize(SeqStack *stack)
        {
            if (NULL == stack) return 0;
            return (stack->top + 1);
        }
        \end{minted}

    \item 获取栈最大容量 \mintinline{c}{int StackCapacity(SeqStack *stack);}
        \begin{minted}{c}
        int StackCapacity(SeqStack *stack)
        {
            if (NULL == stack) return 0;
            return stack->capacity;
            /* 对于静态分配的栈 */
            //return MaxSize;
        }
        \end{minted}

    \item 判断栈是否为空 \mintinline{c}{bool StackEmpty(SeqStack *stack);}
        \begin{minted}{c}
        bool StackEmpty(SeqStack *stack)
        {
            if (NULL == stack) return true;
            return (0 == StackSize(stack));
            /* 或者 */
            //return  (-1 == stack->top);
        }
        \end{minted}

    \item 判断栈是否为满 \mintinline{c}{bool StackFull(SeqStack *stack);}
        \begin{minted}{c}
        bool StackFull(SeqStack *stack)
        {
            if (NULL == stack) return false;
            return (StackSize(stack) == StackCapacity(stack));
            /* 或者 */
            //return (stack->top +1 == stack->capacity);
        }
        \end{minted}

    \item 进栈 \mintinline{c}{bool Push(SeqStack *stack, int ele);}
        \begin{minted}{c}
        bool Push(SeqStack *stack, int ele)
        {
            if (NULL == stack || StackFull(stack)) return false;
            stack->top += 1;
            stack->data[stack->top] = ele;
            return true;
        }
        \end{minted}

    \item 出栈 \mintinline{c}{bool Pop(SeqStack *stack, int *store);}
        \begin{minted}{c}
        bool Pop(SeqStack *stack, int *store)
        {
            if (NULL == stack || NULL == store) return false;
            if (StackEmpty(stack)) return false;
            *store = stack->data[stack->top];
            stack->top -= 1;
            return true;
        }
        \end{minted}

    \item 获取栈顶元素 \mintinline{c}{bool GetTop(SeqStack *stack, int *store);}
        \begin{minted}{c}
        bool GetTop(SeqStack *stack, int *store)
        {
            if (NULL == stack || NULL == store) return false;
            if (StackEmpty(stack)) return false;
            *store = stack->data[stack->top];
            return true;
        }
        \end{minted}

\end{enumerate}

\subsubsection{链式存储的栈}
\begin{minted}{c}
typedef struct LinkStack {
    int size;   /* 存储了多少个元素 */
    LinkList top;
} LinkStack;
\end{minted}

\paragraph{基本操作}
\begin{enumerate}
    \item 初始化栈 \mintinline{c}{bool InitStack(LinkStack *stack);}
        \begin{minted}{c}
        bool InitStack(LinkStack *stack)
        {
            if (NULL === stack) return false;
            stack->size = 0;
            stack->top = NULL;  /* 不带头节点的单链表 */
            return true;
        }
        \end{minted}

    \item 获取栈内元素个数、判断栈是否为空等函数不再赘述

    \item 进栈 \mintinline{c}{bool Push(LinkStack *stack, int ele);}
        \inputminted{c}{codes/push-into-link-stack.c}

    \item 出栈 \mintinline{c}{bool Pop(LinkStack *stack, int *store);}
        \inputminted{c}{codes/pop-from-link-stack.c}

    \item 获取栈顶元素 \mintinline{c}{int GetElem(LinkStack *stack);} 无须再给出吧
\end{enumerate}

\subsection{例题}
栈的应用更多是在数和图里面作为辅助数据结构使用,所以单独考察变体不多。这里以判断括号是否匹配为例子,演示栈数据结构的使用。注意,这里栈内存储元素类型为 \mintinline{c}{char},而不是前文示例所讲的元素类型 \mintinline{c}{int},也就是说下面函数中的栈需要做适当的修改。\label{is-brackets-valid}
\inputminted{c}{codes/is-brackets-valid.c}


\section{队列}
\subsection{数据结构定义}
\subsubsection{顺序存储的队列}
同顺序表和顺序存储的栈,顺序存储的队列也有两种结构。下文只介绍动态分配的顺序存储队列的实现。
\paragraph{动态分配}
\begin{minted}{c}
typedef struct SeqQueue {
    int *data;
    int capacity;
    int front, rear;
    /* 通过 size 记录存储元素的个数,进而作为另一种判断队列是否为空的方法 */
    // int size;
} SeqQueue;
\end{minted}

\paragraph{基本操作}
\begin{enumerate}
    \item 初始化队列 \mintinline{c}{bool InitQueue(SeqQueue *queue, int capacity);}
        \inputminted{c}{codes/init-sequence-queue.c}

    \item 判断队列是否为空 \mintinline{c}{bool QueueEmpty(SeqQueue *queue);}
        \begin{minted}{c}
        bool QueueEmpty(SeqQueue *queue)
        {
            if (NULL == queue) return true;
            return (queue->front == queue->rear);
            /* 如果采用 size 记录存储元素的个数 */
            //return (0 == queue->size);
        }
        \end{minted}

    \item 判断队列是否为满 \mintinline{c}{bool QueueFull(SeqQueue *queue);}
        \begin{minted}{c}
        bool QueueFull(SeqQueue *queue)
        {
            if (NULL == queue) return false;
            return (queue->front == (queue->rear+1) % queue->capacity);
            //return (queue->capacity == queue->size);
        }
        \end{minted}

    \item 获取队列当前元素个数 \mintinline{c}{int QueueSize(SeqQueue *queue);}
        \begin{minted}{c}
        int QueueSize(SeqQueue *queue)
        {
            if (NULL == queue) return 0;
            return (queue->rear - queue->front + queue->capacity) % queue->capacity;
            //return queue->size;
        }
        \end{minted}

    \item 入队 \mintinline{c}{bool EnQueue(SeqQueue *queue, int ele);}
        \begin{minted}{c}
        bool EnQueue(SeqQueue *queue, int ele)
        {
            if (NULL == queue || QueueFull(queue)) return false;
            queue->data[queue->rear] = ele;
            queue->rear = (queue->rear+1) % queue->capacity;
            //queue->size += 1;
            return true;
        }
        \end{minted}

    \item 出队 \mintinline{c}{bool DeQueue(SeqQueue *queue, int *store);}
        \begin{minted}{c}
        bool DeQueue(SeqQueue *queue, int *store)
        {
            if (NULL == queue || NULL == store) return false;
            if (QueueEmpty(queue)) return false;
            *store = queue->data[queue->front];
            queue->front = (queue->front+1) % queue->capacity;
            //queue->size -= 1;
            return true;
        }
        \end{minted}

    \item 获取队首元素 \mintinline{c}{bool GetHead(SeqQueue *queue, int *store);}
        \begin{minted}{c}
        bool GetHead(SeqQueue *queue, int *store)
        {
            if (NULL == queue || NULL == store) return false;
            if (QueueEmpty(queue)) return false;
            *store = queue->data[queue->front];
            return true;
        }
        \end{minted}

\end{enumerate}

\subsubsection{链式存储的队列}
单链表实现的队列,如果需要求得当前队列元素个数,必须要遍历队列,效率较低,也可以通过一个变量 \mintinline{c}{size} 记录存储元素的个数。
\begin{minted}{c}
typedef struct LinkQueue {
    LinkList rear, front;
    int size;
} LinkQueue;
\end{minted}

\paragraph{基本操作}
\begin{enumerate}
    \item 初始化队列 \mintinline{c}{bool InitQueue(LinkQueue *queue);}
        \inputminted{c}{codes/init-link-queue.c}

    \item 销毁队列 \mintinline{c}{void DestroyQueue(LinkQueue *queue);}
        \begin{minted}{c}
        void DestroyQueue(LinkQueue *queue)
        {
            if (NULL == queue) return;
            DestroyList(queue->front);
        }
        \end{minted}

    \item 判断队列是否为空 \mintinline{c}{bool QueueEmpty(LinkQueue *queue);}
        \begin{minted}{c}
        bool QueueEmpty(LinkQueue *queue)
        {
            if (NULL == queue) return true;
            return (queue->front == queue->rear);
        }
        \end{minted}

    \item 入队 \mintinline{c}{bool EnQueue(LinkQueue *queue, int ele);}
        \inputminted{c}{codes/enqueue-link-queue.c}

    \item 出队 \mintinline{c}{bool DeQueue(LinkQueue *queue, int *store);}
        \inputminted{c}{codes/dequeue-link-queue.c}

\end{enumerate}

\subsection{例题}
后续章节树和图的遍历会大量使用队列和栈,所以这里不再单独练习。


\section{树}
\subsection{数据结构定义}
\subsubsection{双亲存储结构}
\paragraph{数组形式}
数组形式存储的 $k\ (k \ge 2)$ 叉树,通常孩子节点下标 $n$ 和双亲节点下标 $m$ 满足如下关系(下标从 1 开始计数):
\begin{itemize}
    \item 通过孩子节点下标计算双亲节点下标:$m = \left\lceil \dfrac{n-1}{k} \right\rceil$
    \item 计算双亲节点的第 $i\ (1 \le i \le k)$ 个孩子的下标: $n = (m - 1)k+i+1$
\end{itemize}
下面证明如何通过孩子节点计算双亲节点公式正确:
\begin{proof}
    对于任意孩子节点 $n$,假设其双亲节点为 $m$。那么双亲 $m$ 的第一个孩子节点下标为 $(m-1) k+2$,最后一个孩子节点下标为 $mk+1$,那么有不等式
    $$
    \begin{aligned}
    & (m-1)k + 2 \le n \le mk+1 \\
        \implies & m -1 + \frac{2}{k} \le \frac{n}{k} \le m+\frac{1}{k} \\
        \implies & m+ \frac{1}{k} - 1 \le \frac{n-1}{k} \le m \\
        \implies & m \le \left\lceil \frac{n-1}{k} \right\rceil \le m \\
        \implies & m = \left\lceil \frac{n-1}{k} \right\rceil
    \end{aligned}
    $$
\end{proof}
特别地,对于二叉树上述公式也是成立,但是此外有更为便捷的方法。考虑到对计算机而言,下取整比上取整更为方便,对上述证明中采取如下计算方法:
$$
\begin{aligned}
& (m-1)k + 2 \le n \le mk+1 \\
    \implies & m \le \frac{n}{k} + 1 - \frac{2}{k} \le m + 1 - \frac{1}{k} \\
    \implies & m \le \left\lfloor \frac{n}{k} + 1 - \frac{2}{k} \right\rfloor \le \left\lfloor m + 1-\frac{1}{k} \right\rfloor = m \\
    \implies & m = \left\lfloor \frac{n}{k} + 1 - \frac{2}{k} \right\rfloor (\text{考虑到 } k = 2) \\
    \implies & m = \left\lfloor \frac{n}{2} \right\rfloor \\
\end{aligned}
$$
即由双亲节点得孩子节点 $n = 2m$ 或 $2m + 1$,由孩子节点得双亲节点 $m = \left\lfloor \dfrac{n}{2} \right\rfloor$

\begin{minted}{c}
#define TreeNodeMax 2048
/* Parent Tree Node */
typedef struct PTNode {
    int data;
    int parent;     /* -1 表明没有双亲节点 */
} PTNode;
/* Parent Tree */
typedef struct PTree {
    PTNode *nodes;
    int capacity;   /* nodes 的总个数 */
} PTree;
\end{minted}

\paragraph{应用}
\begin{enumerate}
    \item 并查集 (Union Find Set):并查集使用来判断两个元素是否在同一个集合的数据结构。
        顺序存储的树结构可以用于实现并查集,不同之处在于并查集不需要通过双亲节点找到孩子节点,因此可以更为灵活。注意:下列代码中空集定义为双亲元素为 -1,单元素集合定义为双亲元素为自身。
        \begin{itemize}
            \item 结构的定义
                \begin{minted}{c}
                typedef struct UFSet {
                    int *nodes;
                    int capacity;
                } UFSet;
                \end{minted}

            \item 初始化并查集 \mintinline{c}{bool InitUFSet(UFSet *set, int capacity);}
                \inputminted{c}{codes/init-ufset.c}

            \item 销毁并查集 \mintinline{c}{void DestroyUFSet(UFSet *set);}
                \begin{minted}{c}
                void DestroyUFSet(UFSet *set)
                {
                    if (NULL == set) return;
                    free(set->nodes);
                }
                \end{minted}

            \item 添加单元素集合 \{id\} \mintinline{c}{bool MakeSet(UFSet *set, int id);}
                \begin{minted}{c}
                bool MakeSet(UFSet *set, int id)
                {
                    if (NULL == set) return false;
                    if (id < 0 || id >= set->capacity) return false;
                    /* 元素 id 已经添加 */
                    if (set->nodes[id] != -1) return false;
                    set->nodes[id] = id;
                    return true;
                }
                \end{minted}

            \item 查找元素 \mintinline{c}{id} 所在集合的代表元素 \mintinline{c}{int Find(UFSet *set, int id);}
                \inputminted{c}{codes/ufset-find.c}

            \item 判断两个元素是否同一集合 \mintinline{c}{bool Same(UFSet *set, int x, int y);}
                \begin{minted}{c}
                bool Same(UFSet *set, int x, int y)
                {
                    int xparent = Find(set, x);
                    int yparent = Find(set, y);
                    if (-1 == xparent || -1 == yparent)
                        return false;
                    return (xparent == yparent);
                }
                \end{minted}

            \item 合并 \mintinline{c}{x} 和 \mintinline{c}{y} 所在的集合 \mintinline{c}{bool Union(UFSet *set, int x, int y);}
                下面单纯的合并算法会导致查找函数 \mintinline{c}{Find} 效率降低,实际中有两种方法优化:按秩合并和路径压缩的方法。优化方法具体实现请参见\footnote{\url{https://en.wikipedia.org/wiki/Disjoint-set_data_structure}}。
                \inputminted{c}{codes/ufset-union.c}

        \end{itemize}
    \item 存储满二叉树
    \item 大堆或小堆 % TODO 实现大堆和小堆
\end{enumerate}

\paragraph{指针形式} 指针形式通常不能直接找到双亲节点的孩子节点,通过孩子节点寻找双亲节点很方便。
\begin{minted}{c}
typedef struct PTreeNode {
    int data;
    struct PTreeNode *parent;
} PTreeNode;
typedef PTreeNode *PTree;
\end{minted}

\subsubsection{孩子兄弟存储结构}
\begin{minted}{c}
/* Child Sibling Tree */
typedef struct CSTreeNode {
    int data;
    struct CSTreeNode *firstchild, *nextsibling;
} CSTreeNode;
typedef CSTreeNode *CSTree;
\end{minted}

\subsubsection{二叉树、二叉排序数和平衡二叉树}
%\paragraph{二叉树结构定义}
\begin{minted}{c}
/* Binary Tree */
typedef struct BiTreeNode {
    int data;
    struct BiTreeNode *lchild, *rchild;
} BiTreeNode;
typedef BiTreeNode *BiTree;
\end{minted}

\paragraph{基本操作}
\subparagraph{二叉树遍历}
\begin{enumerate}
    \item 先序遍历 \mintinline{c}{void PreOrder(BiTree tree);}
        \begin{itemize}
            \item 递归实现
                \begin{minted}{c}
                void PreOrder(BiTree tree)
                {
                    if (NULL == tree) return;
                    /* 访问节点 tree */
                    //visit(tree);
                    PreOrder(tree->lchild);
                    PreOrder(tree->rchild);
                }
                \end{minted}
            \item 非递归实现
                \begin{itemize}
                    \item[$\circ$] 版本一
                        \inputminted{c}{codes/preorder-of-bitree-without-recursion1.c}
                    \item[$\circ$] 版本二*
                        \inputminted{c}{codes/preorder-of-bitree-without-recursion2.c}
                \end{itemize}
        \end{itemize}

    \item 中序遍历 \mintinline{c}{void InOrder(BiTree tree);}
        \begin{itemize}
            \item 递归实现
                \begin{minted}{c}
                void InOrder(BiTree tree)
                {
                    if (NULL == tree) return;
                    InOrder(tree->lchild);
                    /* 访问节点 tree */
                    //visit(tree);
                    InOrder(tree->rchild);
                }
                \end{minted}

            \item 非递归实现
                \inputminted{c}{codes/inorder-of-bitree-without-recursion.c}

        \end{itemize}

    \item 后序遍历 \mintinline{c}{void PostOrder(BiTree tree);}
        \begin{itemize}
            \item 递归实现
                \begin{minted}{c}
                void PostOrder(BiTree tree)
                {
                    if (NULL == tree) return;
                    PostOrder(tree->lchild);
                    PostOrder(tree->rchild);
                    /* 访问节点 tree */
                    //visit(tree);
                }
                \end{minted}

            \item 非递归实现 \footnote{\url{https://en.wikipedia.org/wiki/Tree_traversal\#Post-order_(LRN)}}
                \inputminted{c}{codes/postorder-of-bitree-without-recursion.c}
        \end{itemize}

    \item 层次遍历 \mintinline{c}{void LevelOrder(BiTree tree);}
        \inputminted{c}{codes/levelorder-of-bitree.c}

\end{enumerate}

\subparagraph{二叉查找树} 二叉查找树 (Binary Search Tree, BST),也叫排序二叉树或二叉排序树
\begin{enumerate}
    \item 查找元素 \mintinline{c}{bool BiTreeSearch(BiTree tree, int data);}
        \begin{itemize}
            \item 递归实现
                \begin{minted}{c}
                bool BiTreeSearch(BiTree tree, int data)
                {
                    if (NULL == tree) return false;
                    if (data == tree->data)
                        return true;
                    if (data < tree->data)
                        return BiTreeSearch(tree->lchild, data);
                    else
                        return BiTreeSearch(tree->rchild, data);
                }
                \end{minted}

            \item 非递归实现
                \begin{minted}{c}
                bool BiTreeSearch(BiTree tree, int data)
                {
                    while (NULL != tree) {
                        if (data == tree->data)
                            return true;
                        if (data < tree->lchild)
                            tree = tree->lchild;
                        else
                            tree = tree->rchild;
                    }

                    return false;
                }
                \end{minted}

        \end{itemize}

    \item 插入元素 \mintinline{c}{bool BiTreeInsert(BiTree *ptree, int data);}
        \inputminted{c}{codes/bst-insert.c}

    \item 二叉排序树构造 \mintinline{c}{BiTree CreateBSTree(int *array, int n);}
        \begin{minted}{c}
        BiTree CreateBSTree(int *array, int n)
        {
            if (NULL == array || n <= 0) return NULL;

            BiTree tree = NULL;
            for (int i = 0; i < n; i++)
                BiTreeInsert(&tree, array[i]);

            return tree;
        }
        \end{minted}

    \item 删除元素* \mintinline{c}{bool BiTreeDelete(BiTree *ptree, int data);}
        \inputminted{c}{codes/bst-delete.c}

\end{enumerate}

\subparagraph{平衡二叉树} 平衡二叉树 AVL-Tree,查找算法和二叉查找树一样,但是插入和删除算法不同而且代码实现较为复杂(比二叉查找树删除元素还要复杂),这里暂时不给出。

\paragraph{应用}
\subparagraph{实现集合*}


\subsubsection{线索二叉树}
\begin{minted}{c}
typedef struct ThreadTreeNode {
    int data;
    struct ThreadTreeNode *lchild, *rchild;
#define TagChild    0
#define TagPrevNode 1
#define TagNextNode 1
    int ltag, rtag;
} ThreadTreeNode;
/**
 * ltag 为 TagChild    (0) 时,lchild 指向左孩子节点
 * ltag 为 TagPrevNode (1) 时,lchild 指向前驱节点
 * rtag 为 TagChild    (0) 时,rchild 指向右孩子结点
 * rtag 为 TagNextNode (1) 时,rchild 指向后继节点
 */
 typedef ThreadTreeNode *ThreadTree;
\end{minted}
线索化的思想:任意一个节点有前驱节点且左孩子指针 \mintinline{c}{lchild} 没有指向左孩子,那么就指向此节点的前驱节点;任意一个节点有后继节点且右孩子指针 \mintinline{c}{rchild} 没有指向右孩子,那么就指向后继节点。

\paragraph{基本操作}
\begin{enumerate}
    \item 先序遍历线索化* \mintinline{c}{void CreatePreThreadTree(ThreadTree tree);}
        \inputminted{c}{codes/create-preorder-thread-tree.c}

    \item 中序遍历线索化* \mintinline{c}{void CreateInThreadTree(ThreadTree tree);}
        \inputminted{c}{codes/create-inorder-thread-tree.c}

    \item 后序遍历线索化* \mintinline{c}{void CreatePostThreadTree(ThreadTree tree);}
        \inputminted{c}{codes/create-postorder-thread-tree.c}

    \item 先序遍历先序线索树 \mintinline{c}{void PreOrderOfPreThreadTree(ThreadTree tree);}
        \inputminted{c}{codes/preorder-of-prethread-tree.c}

    \item 中序遍历中序线索树 \mintinline{c}{void InOrderOfInThraedTree(ThreadTree tree);}
        \inputminted{c}{codes/inorder-of-inthread-tree.c}
        对中序线索树中序遍历时,某一节点的前驱或者后继节点一定是其\emph{子树}中的节点,所以一定方便找到前驱和后继。对比先序线索树的先序遍历,某一节点的前去节点可能是其\emph{父节点},所以难以找到前驱节点。

    \item 后序遍历后序线索树 \mintinline{c}{void PostOrderOfPostThread(ThreadTree tree);}
        \inputminted{c}{codes/postorder-of-postthread-tree.c}
        对比先序线索树的先序遍历,可以发现后序线索树的后续遍历与其有一定的对称性。
\end{enumerate}

\subsection{例题}
\begin{enumerate}
    \item 求二叉树的高度
        \inputminted{c}{codes/height-of-bitree.c}

    \item 求二叉树宽度和高度,二叉树宽度定义为同一层中节点数最大值
        \inputminted{c}{codes/height-and-width-of-bitree.c}

    \item 求左子树中节点经过根节点到右子树中节点的最长路径的长度
        \inputminted{c}{codes/max-distance-through-root.c}

    \item 输出根节点到每个叶子结点的路径
        \inputminted{c}{codes/print-all-traces.c}

    \item 输出二叉树给定节点的所有祖先
        \inputminted{c}{codes/print-ancestors.c}

    \item 判断给定二叉树是否为完全二叉树
        \inputminted{c}{codes/is-complete-bitree.c}

    \item 判断二叉树 \mintinline{c}{subtree} 是否为 \mintinline{c}{tree} 的一棵\emph{子树}
        \inputminted{c}{codes/is-subtree.c}

    \item 判断二叉树 \mintinline{c}{part} 是否为 \mintinline{c}{tree} 的\emph{子结构},子结构是指能在 \mintinline{c}{tree} 找到 \mintinline{c}{part} 这样的树结构,并不一定是子树。
        \inputminted{c}{codes/is-part-of-bitree.c}

    \item 按照从下往上、从右往左的方式遍历二叉树(逆层次遍历)
        \inputminted{c}{codes/re-level-order.c}

    \item 通过先序序列和中序序列重构二叉树*
        \inputminted{c}{codes/rebuild-bitree.c}

    \item 由满二叉树的先序序列求出其后序序列,也就是对顺序存储的满二叉树数组进行后续遍历
        \inputminted{c}{codes/from-preorder-to-postorder.c}

    \item 统计二叉树的节点数、叶子数、双分叉节点数
        \inputminted{c}{codes/statistic-of-bitree.c}

    \item 交换左右子树
        \inputminted{c}{codes/swap-bitree.c}

    \item 删除二叉树所有值为 \mintinline{c}{x} 的\emph{子树}(不是节点)
        \inputminted{c}{codes/delete-all-subtree-x-from-bitree.c}

    \item 删除二叉树(不是二叉排序树)所有值为 \mintinline{c}{x} 的\emph{节点}*,由于不是二叉排序树,所以删除节点后的树不唯一,只需要保证是二叉树即可
        \inputminted{c}{codes/delete-all-x-from-bitree.c}

    \item 查找二叉树节点 \mintinline{c}{p} 和 \mintinline{c}{q} 的最近公共祖先
        \inputminted{c}{codes/common-ancestor.c}

    \item 将二叉树的叶子节点从左向右顺序连接,即把所有叶子节点组织成非循环双链表
        \inputminted{c}{codes/link-leafs-of-bitree.c}

    \item 判断两棵二叉树结构是否相似(只判断形状不判断节点值)
        \inputminted{c}{codes/is-same-bitree.c}

    \item 在中序线索树查找给定节点在\emph{后序遍历序列}中的前驱节点

        思想:若此节点有右子树,那么前驱就是右节点;若有左子树,那么就是左节点;如果左右子树都没有,那么节点左指针指向某一个祖先节点,如果双亲节点有左子树,那么前驱就是祖先节点的左节点,如果左节点没有则继续往上寻找,知道某一个祖先节点满足具有左子树,除非左子树为空指针,那么这个节点是后序遍历的第一个节点,没有前驱。
        \inputminted{c}{codes/find-previous-node-of-postorder-in-inthreadtree.c}

    \item 计算二叉树的带权路径长度 (Weighted Path Length, WPL)
        \inputminted{c}{codes/weighted-path-length-of-bitree.c}

    \item 输出给定表达树的中缀表达式子(通过括号体现优先级)
        \inputminted{c}{codes/print-expression-bitree.c}

    \item 求指定值在二叉排序树中的高度
        \inputminted{c}{codes/height-of-value-in-bsttree.c}

    \item 判断给定二叉树是否是\emph{二叉排序树}
        \inputminted{c}{codes/is-bsttree.c}


    \item 判断给定二叉树是否是\emph{平衡二叉树}
        \inputminted{c}{codes/is-avl-tree.c}

    \item 求二叉排序树的最大值和最小值
        \inputminted{c}{codes/max-and-min-value-in-bsttree.c}

    \item \emph{从大到小} 输出二叉排序树中大于等于 \mintinline{c}{lo} 小于等于 \mintinline{c}{hi} 的节点的值
        \begin{itemize}
            \item 版本一:遍历整棵树,根据判断是否输出节点值
                \inputminted{c}{codes/print-from-hi-to-lo-in-bsttree1.c}
            \item 版本二:仅仅遍历要输出的节点
                \inputminted{c}{codes/print-from-hi-to-lo-in-bsttree2.c}
        \end{itemize}

    \item 以 $O(\log_2(n))$ 时间复杂度查找二叉树(假定二叉树树没有退化成链表)中序遍历序列中第 $k\ (1 \le k \le n)$ 个节点。二叉树节点中有附加域 \mintinline{c}{nodes} 表明当前子树的节点个数,比如空树没有节点,所以无法存储 \mintinline{c}{nodes},就认为空树节点数为 0;只有一个节点的子树其根节点 \mintinline{c}{nodes} 为 1;其他树的根节点的 \mintinline{c}{nodes} 等于左右子树的 \mintinline{c}{nodes} 之和再加 1 \label{find-kth-node-in-bitree}
        \inputminted{c}{codes/find-kth-node-in-bitree.c}

\end{enumerate}


\section{广义表 *}
\subsection{数据结构定义}
广义表的字段 \mintinline{c}{isAtom} 用于表明此节点是一个\emph{原子}节点还是普通节点。当 \mintinline{c}{isAtom} 为真时,此节点为原子节点,此时 \mintinline{c}{left} 和 \mintinline{c}{right} 字段没有意义;同理,当 \mintinline{c}{isAtom} 为假时,此节点为普通节点 \mintinline{c}{data} 字段没有意义\footnote{在 C 语言中更常用联合体 \mintinline{c}{union} 来表示这种结构}。
\begin{minted}{c}
/* General List Node */
typedef struct GListNode {
    int data;
    struct GListNode *left, *right;
    bool isAtom;
} GListNode;
/* General List */
typedef GListNode *GList;
\end{minted}

\paragraph{基本操作}
\begin{enumerate}
    \item 判断一个节点是不是原子 \mintinline{c}{bool IsAtom(GList list);}
        \begin{minted}{c}
        bool IsAtom(GList list)
        {
            /* 空列表不是原子 */
            if (NULL == list) return false;
            return list->isAtom;
        }
        \end{minted}

    \item 判断一个节点是不是“节点对” (pair) \mintinline{c}{bool IsPair(GList list);}
        \begin{minted}{c}
        bool IsPair(GList list)
        {
            /* 空列表不是节点对 */
            if (NULL == list) return false;
            return !IsAtom(list);
        }
        \end{minted}

    \item 判断一个列表是不是空列表 \mintinline{c}{bool IsNull(GList list);}
        \begin{minted}{c}
        bool IsNull(GList list)
        {
            return (NULL == list);
        }
        \end{minted}

    \item 根据数据构建原子节点 \mintinline{c}{GList MakeAtom(int data);}
        \begin{minted}{c}
        GList MakeAtom(int data)
        {
            GList node = malloc(sizeof(GListNode));
            if (NULL == node) return NULL;
            node->data = data;
            node->isAtom = true;
            return node;
        }
        \end{minted}

    \item 构建对 (pair) \mintinline{c}{GList MakePair(GList a, GList b);} 或者命名为 \mintinline{c}{GList cons(GList a, GList b);}。想知道为何是 \mintinline{c}{cons} 见 \footnote{\url{https://en.wikipedia.org/wiki/Lisp_(programming_language)\#Conses_and_lists}} \footnote{\url{https://en.wikipedia.org/wiki/Cons}}。
        \begin{minted}{c}
        GList MakePair(GList a, GList b)
        {
            GList node = malloc(sizeof(GListNode));
            if (NULL == node) return NULL;
            node->left = a;
            node->right = b;
            node->isAtom = false;
            return node;
        }
        \end{minted}

    \item 构建列表 \mintinline{c}{GList MakeList(int *arr, int len);}把数组中的元素构建成列表
        \begin{minted}{c}
        GList MakeList(int *arr, int len)
        {
            if (NULL == arr || len <= ) return NULL;    /* 空列表 */
            GList list = NULL;
            for (int i = len-1; i >= 0; i--)
                list = MakePair(MakeAtom(arr[i]), list);
            return list;
        }
        \end{minted}

    \item 取表头 \mintinline{c}{GList GetHead(GList list);} 或者命名为 \mintinline{c}{GList car(GList list);}
        \begin{minted}{c}
        GList GetHead(GList list)
        {
            if (IsNull(list)) return NULL;  /* 空表表头还是空表 */
            if (IsAtom(list)) return NULL;  /* 原子没法取“表头” */
            return list->left;
        }
        \end{minted}

    \item 取除去表头剩下的部分 \mintinline{c}{GList GetTail(GList list);} 或者命名为 \mintinline{c}{GList cdr(GList list);}
        \begin{minted}{c}
        GList GetTail(GList list)
        {
            if (IsNull(list)) return NULL;  /* 空表表尾还是空表 */
            if (IsAtom(list)) return NULL;
            return list->right;
        }
        \end{minted}


    \item 判断是不是列表 \mintinline{c}{bool IsList(GList list);}
        \begin{minted}{c}
        bool IsList(GList list)
        {
            if (IsNull(list)) return true;  /* 空表 */
            /* 顺着列表往后找到最后一个非原子节点 */
            while (IsPair(list))
                list = GetTail(list);
            /* 这个非原子节点必须是空表,那么才是列表 */
            return IsNull(list);
        }
        \end{minted}

\end{enumerate}

\subsection{例题}
\begin{enumerate}
    \item 构建广义列表 \verb|list = (A, (G, H, (M)), D)| \label{build-glist}
        \begin{minted}{c}
        GList listM = MakePair(MakeAtom('M'), NULL);    /* (M) */
        GList listHM = MakePair(MakeAtom('H'), listM);  /* (H, (M)) */
        GList listGHM = MakePair(MakeAtom('G'), listHM);/* (G, H, (M)) */
        GList listD = MakePair(MakeAtom('D'), NULL);    /* (D) */
        GList listGHMD = MakePair(listGHM, listD);      /* ((G, H, (M)), D) */
        GList list = MakePair(MakeAtom('A'), listGHMD); /* (A, (G, H, (M)), D) */
        \end{minted}

    \item 在广义表 \verb|list = (A, B, (F, C), D)| 中获取 \verb|C|
        \begin{minted}{c}
        GetTail(list);                                     /* (B, (F, C), D) */
        GetTail(GetTail(list));                            /* ((F, C), D) */
        GetHead(GetTail(GetTail(list)));                   /* (F, C) */
        GetTail(GetHead(GetTail(GetTail(list))));          /* (C) */
        GetHead(GetTail(GetHead(GetTail(GetTail(list))))); /* C */
        \end{minted}

    \item 求广义表的深度,即 \mintinline{c}{left} 嵌套的深度,直观来说就是左括号 \verb|(| 的嵌套的最大深度。比如本章第~\ref{build-glist}~题表深度为 3
        \begin{minted}{c}
        int DepthOfGList(GList list)
        {
            if (IsNull(list) || IsAtom(list)) return 0;
            int leftDepth = DepthOfGList(GetHead(list));
            int rightDepth = DepthOfGList(GetTail(list));
            return max(1 + leftDepth, rightDepth);
        }
        \end{minted}

    \item 求广义表的长度,比如本章第~\ref{build-glist}~题表长度为 3
        \begin{minted}{c}
        int LengthOfGList(GList list)
        {
            if (!IsList(list)) return 0;
            int len = 0;
            while (!IsNull(list)) {
                len ++;
                list = GetTail(list);
            }
            return len;
        }
        \end{minted}

\end{enumerate}


\section{图}
\subsection{数据结构定义}

\paragraph{遍历}
\begin{enumerate}
    \item 深度优先遍历
    \item 广度优先遍历
\end{enumerate}

\subsection{例题}


\section{排序和查找}

\subsection{排序}

\subsubsection{插入排序}
\inputminted{c}{codes/insert-sort.c}

\subsubsection{折半插入排序}
仅仅在查找元素的时候采用二分查找,但是移动元素开销不能避免,同时代码也会比简单的插入排序复杂一点点,所以不给出代码。

\subsubsection{希尔排序*}
希尔排序实际上是插入排序的范化版本,插入排序可以看成增量固定为1的希尔排序。希尔排序像是多次使用插入排序,每次插入排序的增量不断减小直到为1.
\begin{itemize}
    \item 以增量减半作为增量序列的希尔排序
        \inputminted{c}{codes/shell-sort.c}
    \item 执行指定增量序列的希尔排序
        \inputminted{c}{codes/shell-sort-specified-incs.c}
\end{itemize}

\subsubsection{冒泡排序}
\inputminted{c}{codes/bubble-sort.c}

\subsubsection{快速排序}
\inputminted{c}{codes/quick-sort.c}
\paragraph{应用}
快速排序中 \mintinline{c}{Partition} 是一个重要的想法。通过这个方法可以做到高效地求得无序数组中第 $k$ 大的元素,采取的方法类似于树中第\ref{find-kth-node-in-bitree}题。
\begin{enumerate}
    \item 求无序数组中第 $k\ (1 \le k \le n)$ 大的元素。
        \inputminted{c}{codes/find-kth-in-unsort-array.c}
\end{enumerate}


\subsubsection{选择排序}
\inputminted{c}{codes/select-sort.c}

\subsubsection{堆排序*}
\inputminted{c}{codes/heap-sort.c}

\subsubsection{归并排序}
\begin{enumerate}
    \item 递归版本实现,也是最常见的实现方式
        \inputminted{c}{codes/merge-sort1.c}
    \item 非递归实现*,从底向上合并的思想
        \inputminted{c}{codes/merge-sort2.c}
\end{enumerate}

\subsubsection{基数排序*}
基数排序只能用于特定类型的数据的排序,比如整数或字符串,并不是一个通用型排序算法,并且实现需要基数个队列。
\inputminted{c}{codes/radix-sort.c}

\subsection{查找}

\subsubsection{顺序查找}
顺序查找一般用于无序的数组内查找元素。
\begin{minted}{c}
/* 查找并返回元素下标 */
int SequenceSearch(int *arr, int len, int x)
{
    if (NULL = arr || len <= 0) return -1;
    for (int i = 0; i < len; i++) {
        if (x == arr[i])
            return i;
    }
    return -1;
}
\end{minted}

\subsubsection{二分查找}
对于已经有序的数组,二分查找是更为高效的算法,一般假定数据从小到大排序。
\inputminted{c}{codes/binary-search.c}


\end{document}
